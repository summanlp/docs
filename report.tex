\documentclass{llncs}
\usepackage{makeidx}
\usepackage[spanish]{babel}
\usepackage[utf8]{inputenc}
\usepackage{url}
\urldef{\mails}\path|{fbarrios,fjlopez}@fi.uba.ar| 


\begin{document}

\frontmatter
\pagestyle{headings}

\title{Variantes de TextRank para la Generación de Resúmenes Automáticos}
\titlerunning{Variantes de TextRank} 

\author{Federico Barrios \and Federico López}
\institute{Facultad de Ingeniería, Universidad de Buenos Aires,\\
Avenida Paseo Colón 850, Ciudad Autónoma de Buenos Aires, Argentina\\
\mails\\
\url{http://ingenieria.uba.ar/}}

\maketitle

\begin{abstract}
Lorem ipsum dolor sit amet, consectetur adipiscing elit. Nullam nec dictum 
nibh, vel lacinia lectus. Morbi non mi mattis, accumsan arcu id, venenatis 
eros. Nunc est ex, efficitur hendrerit posuere eget, dapibus at orci. 
Cras feugiat erat ut volutpat blandit. Aliquam erat volutpat. Mauris 
auctor congue libero et aliquet. Etiam justo mi, tempor sit amet feugiat 
ut, tincidunt id augue. Pellentesque quis nibh vel turpis mattis dignissim. 
Duis elementum sem quam, eget dignissim nibh suscipit a. Fusce quis magna 
eu neque efficitur ornare nec sed turpis. Aenean mollis ipsum tristique vestibulum tempus. 
Sed in ornare ipsum. Vivamus purus mi, sollicitudin.
\keywords{variantes, TextRank, generación, resúmenes, automáticos}

\end{abstract}

\section{Introducción}
Foo bar baz

\section{Trabajo previo}
Se ha visto un gran avance en el campo de la generación automática de resúmenes desde finales de 1960 hasta la actualidad. Los métodos tradicionales tienen en cuenta la frecuencia de palabras o frases introductorias para identificar las oraciones más sobresalientes del texto. También, se han desarrollado modelos estadísticos basados en corpus de entrenamiento para combinar varias heurísticas: palabras clave, posición de las oraciones, longitud de las oraciones, frecuencia de palabras y palabras contenidas en los títulos \cite{hovy}. Otros enfoques se basan en la representación del texto en forma de grafo. Las oraciones importantes y los conceptos son las entidades altamente conectadas y, por esto, forman parte del resumen \cite{barzilay}. De igual modo, se ha propuesto analizar la estructura discursiva y extraer las relaciones retóricas entre las diferentes unidades textuales, y así separar las principales de las secundarias para descubrir las unidades que juegan un papel preponderante dentro de la estructura discursiva \cite{marcu}.

En la línea de representación del texto como un grafo conectado, se usan técnicas de Recuperación de Información para identificar oraciones similares y determinar las más importantes, que formarán al resumen final \cite{salton}. El enfoque propuesto, tanto por Mihalcea \& Tarau como por Erkan \& Radev, consiste en utilizar el prestigio de las unidades léxicas (oraciones o palabras) dentro del grafo. Dicha técnica ha sido la adoptada en TextRank.


\section{TextRank}

\subsection{Descripción}
TextRank es un algoritmo no supervisado basado en grafos para realizar resúmenes automáticos extractivos y/u obtener palabras claves de un texto. Fue presentado en 2004 por Rada Mihalcea y Paul Tarau en el paper “TextRank: Bringing Order into Texts”.

El algoritmo aplica una variación de PageRank sobre un grafo especialmente diseñado para la tarea. De esta manera permite explotar la estructura del texto, identificando los conceptos principales, sin necesidad de datos previos de entrenamiento. Debido a que se basa en PageRank, se sirve de la noción del “prestigio” o “recomendación” entre los elementos del grafo. Por este motivo, TextRank puede ser aplicado a cualquier texto, incluso en distintos idiomas, generando un resumen basado sólo en las propiedades intrínsecas del texto.

El algoritmo modela el texto en base a un grafo, y luego, busca crear relaciones significativas (aristas) entre las entidades léxicas (vértices). Dependiendo de la aplicación que se desee dar al algoritmo, las entidades pueden ser palabras, frases, oraciones, párrafos, entre otros. De manera similar, también debe definirse el tipo de relación que se usa para unir los vértices: semántica, contextual, de superposición, y demás.

Los pasos principales que se llevan a cabo son los siguientes:

\begin{enumerate}
\item Identificar las unidades del texto (palabras u oraciones) y agregarlas al grafo como vértices.
\item Identificar relaciones que conectan a estas unidades, y agregarlas al grafo como aristas entre los vértices. Las aristas pueden ser dirigidas o no, y ponderadas o no.
\item Aplicar PageRank para asignarle un puntaje a cada vértice.
\item Ordenar los vértices de acuerdo al puntaje y utilizarlo para armar el resumen de acuerdo a algún criterio.
\end{enumerate}

\subsection{Generación de resúmenes automáticos}
El problema de la extracción de oraciones apunta a identificar las secuencias más representativas del texto. Para este caso, las unidades tomadas para aplicar el algoritmo serán oraciones completas.

Inicialmente, se construye un grafo en base al texto. En este caso, se agrega cada oración como vértice. Para crear las aristas con su peso correspondiente, se debe definir una función de similitud entre dos oraciones dadas. Esta función de similitud será la que dicte cuánto una oración “recomienda” a otra, dado que ambas poseen contenidos similares y abordan los mismos conceptos.
    
La función utilizada formalmente se define de la siguiente manera:

\textit{Sean Si , Sj dos oraciones representadas por un conjunto de n palabras que en Si aparecen como: Si = wi1, wi2,,..., win,. La función de similitud para Si, Sj se define como:}
Similitud(Si,Sj) = |{wk|wk Si wk Sj}|log( |Si|) + log( |Sj|)
    
El resultado de este proceso es el texto representado como un grafo ponderado y altamente conexo. En base a esto se aplica PageRank para priorizar los vértices más relevantes.

Luego de aplicar el algoritmo de priorización, se seleccionan las oraciones con mayor puntaje para incluirlas en el resumen. Finalmente, dichas oraciones se las presenta de acuerdo al orden de aparición en el texto original.


\section{Experimentos}

\subsection{Evaluación}

\subsection{Propuestas}

\subsubsection{Subcadena común}

\subsubsection{Similitud coseno}

\subsubsection{BM25}

\subsection{Resultados}

\section{Conclusiones}

\end{document}
